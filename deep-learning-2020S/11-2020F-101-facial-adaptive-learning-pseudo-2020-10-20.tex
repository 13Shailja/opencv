\documentclass[conference]{IEEEtran} 

%2019 hl ref: https://en.wikibooks.org/wiki/LaTeX/Mathematics#Matrices_and_arrays
\ifCLASSINFOpdf 
\else 
\fi  
\hyphenation{op-tical net-works semi-conduc-tor}


%\usepackage{amsmath} 
%HL 2018-1-21: not sure if IEEE accepts this

\usepackage{graphicx}
\graphicspath{images/}
%HL 2018-1-21 add this image folder

%HL 2020-10-19 add this
\usepackage{algorithm}
\usepackage{algorithmic}


\begin{document}
 
% paper title 

\title{Pseudo Code For Facial \\ 
Adaptive Learning Algorithm}
 
\author{
\IEEEauthorblockN{Harry Li $^{\ddagger}$, Ph.D.}
\IEEEauthorblockA{Computer Engineering Department \\
College of Engineering, San Jose State University\\
San Jose, CA 95192, USA} \\
Email$^{\dagger}$: hua.li@sjsu.edu\\
}
%\IEEEauthorblockA{Research and Development Center for Artificial Intelligence and Automation %$^{\ddagger}$ \\
%CTI Plus Corporation, 3679 Enochs Street\\
%Santa Clara, CA 95051, USA} \\
%Email$^{\dagger}$: hua.li@sjsu.edu\\
%}

% make the title area
\maketitle
\begin{abstract}
This note is a collection of some of Pseudo Code that I have creaed for the on-going research and development work in facial adaptive learning technique.   
\end{abstract}

% no keywords 
\IEEEpeerreviewmaketitle

\section{Adaptive Learning Algorithm}
 
In this note, the algorithm for facial adaptive learning 
is discussed below. 
   
%https://www.math-linux.com/latex-26/faq/latex-faq/article/how-to-write-algorithm-and-pseudocode-in-latex-usepackage-algorithm-usepackage-algorithmic

First, define $I(x,y;t_i)$ as input video frame, which can also be 
denoted as $I_i$, hence, for the classes of positve, negative, and 
anchor image, we introduce $I^p_i$, $I^n_i$ and $I^a_i$ notation 
respectively. Note in the conference papaer on Facenet published by 
google team, the above notation can be written as 
$x^p_i$, $x^n_i$ and $x^a_i$ as well. 

Now, define $P_i(w_i,b_i)$ for i=1, 2 for the execution of
facenet based facial recognition algorithm.   
$P_1(w_i,b_i)$ is for the process with an original 
image dataset $\Omega$, while $P_2(w_i,b_i)$ is for the process with
new added images $I(x,y;t_i) \in {\Omega}_a$. Here we have denoted 
${\Omega}_a$ as the enlarged image dataset. 

The 1st Pseudo Code. 

\begin{algorithm}
\caption{Adaptive Learning}
\begin{algorithmic}
\REQUIRE input video frame $I(x,y;t_i)$  
\ENSURE Start 2 Processes $P_i(w_i,b_i)$ for i=1, 2; 
\WHILE{$P_1(w_i,b_i) \wedge P_2(w_i,b_i)$ }
\IF{$flag\_adaptive$}
\STATE Run $P_2(w_i,b_i)$ with new added images $I(x,y;t_i) \in {\Omega}_a$ 
\IF{$flag\_update$}
\STATE Update $P_1(w_i,b_i)$ 
\ELSE
\STATE Run $P_2(w_i,b_i)$ with new added images $I(x,y;t_i) \in {\Omega}_a$ 
\ENDIF
\ELSE
\STATE Run $P_1(w_i,b_i)$ with original image dataset $\Omega$ 
\ENDIF
\ENDWHILE
\end{algorithmic}
\end{algorithm}

The 2nd Pseudo Code. 
Now the algorithm for anchor image selection. 

\begin{algorithm}
\caption{Selection of anchor $I^a(x,y;t_i)$}
\begin{algorithmic}
\REQUIRE input video frame $I(x,y;t_i)$  
\ENSURE Start 2 Processes $P_i(w_i,b_i)$ for i=1, 2; 
\WHILE{$P_1(w_i,b_i) \wedge P_2(w_i,b_i)$ }
\IF{$flag\_adaptive$}
\STATE run $P_2(w_i,b_i)$ with new added images $I(x,y;t_i) \in {\Omega}_a$ 
\IF{$flag\_update$}
\STATE update $P_1(w_i,b_i)$ 
\ELSE
\STATE run $P_2(w_i,b_i)$ with new added images $I(x,y;t_i) \in {\Omega}_a$ 
\ENDIF
\ELSE
\STATE run $P_1(w_i,b_i)$ with original image dataset $\Omega$ 
\ENDIF
\ENDWHILE
\end{algorithmic}
\end{algorithm}

The 3rd Pesudo Code. 
Now, for new image dataset update

\begin{algorithm}
\caption{Update image dataset ${\Omega}_a$}
\begin{algorithmic}
\REQUIRE input video frame $I(x,y;t_i)$  
\ENSURE Start Processes $P_1(w_i,b_i)$; 
\WHILE{$P_1(w_i,b_i)$ }
\IF{$flag\_falseDetection \wedge flag\_indConformation$ }
\STATE Select $I(x,y;t_i)$ make $ I(x,y;t_i) \in {\Omega}_a$ 
\STATE Update $\Omega \rightarrow {\Omega}_a$ 
\ENDIF
\ENDWHILE
\end{algorithmic}
\end{algorithm}

 

% use section* for acknowledgment
\section*{Acknowledgment}

The author would like to express his appreciation to  
the discussions with the CTI One team members. 

% trigger a \newpage just before the given reference
% number - used to balance the columns on the last page
% adjust value as needed - may need to be readjusted if
% the document is modified later
%\IEEEtriggeratref{8}
% The "triggered" command can be changed if desired:
%\IEEEtriggercmd{\enlargethispage{-5in}}

% references section

% can use a bibliography generated by BibTeX as a .bbl file
% BibTeX documentation can be easily obtained at:
% http://mirror.ctan.org/biblio/bibtex/contrib/doc/
% The IEEEtran BibTeX style support page is at:
% http://www.michaelshell.org/tex/ieeetran/bibtex/
%\bibliographystyle{IEEEtran}
% argument is your BibTeX string definitions and bibliography database(s)
%\bibliography{IEEEabrv,../bib/paper}
%
% <OR> manually copy in the resultant .bbl file
% set second argument of \begin to the number of references
% (used to reserve space for the reference number labels box)
\begin{thebibliography}{1}

%\bibitem{IEEEhowto:kopka}
%H.~Kopka and P.~W. Daly, \emph{A Guide to \LaTeX}, 3rd~ed.\hskip 1em plus
%  0.5em minus 0.4em\relax Harlow, England: Addison-Wesley, 1999.

\bibitem{Waymore}  
Inverse Mapping to World Coordinate System, Self driving cars project internal report. 

\bibitem{B.K.P. Horn}  
\textit{B.K.P. Horn, Robot Vision}.  
MIT Press, 1982. 

\end{thebibliography}




% that's all folks
\end{document}


